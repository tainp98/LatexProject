\documentclass{article}
\usepackage{graphicx}
\usepackage{wrapfig}
\usepackage{placeins}
\graphicspath{{./images/}}
\begin{document}
\renewcommand{\labelenumii}{\arabic{enumi}.\arabic{enumii}}
\renewcommand{\labelenumiii}{\arabic{enumi}.\arabic{enumii}.\arabic{enumiii}}
\renewcommand{\labelenumiv}{\arabic{enumi}.\arabic{enumii}.\arabic{enumiii}.\arabic{enumiv}}
\begin{enumerate}
\item \textbf{Graph Cuts on GPU}
\begin{enumerate}
    \item Push-Relabel algorithm
    
    // describe push-relabel algorithm
    \item Push-Relabel implemented on CUDA
    \begin{enumerate}
        \item Graph construct on CUDA
        
        Symbols A(i), B(i) are respectively gray level values at pixel i 
        corresponding to two parts of the image that are determined to be 
        the area overlapping.

        Formula that calculate weight between 2 adjacency edges taken from graphcut textures 
        paper of Vivek Kwatra Arno Schodl Irfan Essa Greg Turk Aaron Bobick.
        \begin{equation}
            $$ c(u,v) = c(v,u) = A(u)-B(u) + A(v)-B(v) $$
        \end{equation}
        \begin{figure}[h]
            \includegraphics[width=0.5\textwidth]{graphConstruct.png}
            \caption{Figure 1: Graph construction }
            \centering
        \end{figure}

        \item Data representation (unify memory)
        
        We construct a 4-connectivity graph, so we need 4 arrays with each array size equal
        |V|, V is set of vertices of graph G to hold 4 weights for each node corresponding 
        it’s 4 adjacency edges. To access all weights of node i, we access i-th element of 
        all 4 arrays. If the node lacks any adjacency edge, the weight on that side equal 0. 
        One array of size |V| contain excess flow of all vertices and one array of size |V|
        hold heigh of all vertices. Relabel kernel ollow the push kernel. Relabel kernel needs 
        to know which nodes are in need of relabeling, so it is necessary to have a relabel 
        mask array of size |V| to mark the nodes that need relabel.

        \begin{figure}[!h]
            \includegraphics[width=0.5\textwidth]{pushkerneldata.png}
            \caption{Figure 2: Data for push kernel}
        \end{figure}
        \FloatBarrier

        \begin{figure}[!h]
            \includegraphics[width=0.5\textwidth]{relabelkerneldata.png}
            \caption{Figure 2: Data for push kernel}
        \end{figure}
        \FloatBarrier

        \item Push kernel
        \item Relabel kernel
        \item Global relabel CPU
        \item Overall graphcut algorithm
    \end{enumerate}
\end{enumerate}
\item Result and evaluating
\begin{enumerate}
    \item Result
    \item Improved method
    \item Result after using improved method
\end{enumerate}
\end{enumerate}


\end{document}